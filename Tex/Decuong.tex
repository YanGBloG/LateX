\documentclass[12pt,a4paper]{article}
\usepackage[utf8]{inputenc}
\usepackage{vietnam}
\usepackage[left=3cm, right=2cm, top=2.5cm, bottom=2.5cm]{geometry}
\usepackage{amsmath}
%\usepackage{asmfonts}
\usepackage{float}
\usepackage{amssymb} 
\usepackage{graphicx} % thư viện hiển thị hình ảnh
\usepackage[hidelinks, unicode]{hyperref} % thư viện link tới chương
\usepackage{caption} % thư viện đặt caption
\usepackage[table, xcdraw]{xcolor} % thư viện hiển thị màu cho bảng
\usepackage[normalem]{ulem}
%\usepackage{indentfirst}
\usepackage{mathptmx}
\useunder{\uline}{\ul}{}
\renewcommand{\baselinestretch}{1.5}

\newcommand{\subsubsubsection}[1]{\paragraph{#1}\mbox{}\\}

\setcounter{secnumdepth}{4}
\setcounter{tocdepth}{4}
\geometry{letterpaper}

\begin{document}
	\begin{center}
		\textbf{\LARGE{ĐỀ CƯƠNG ĐỒ ÁN KHAI THÁC - DAKT}}
	\end{center}

\section{Tên đồ án:}
Nghiên cứu tối ưu hóa tỉ số độ linh động trong thu hồi dầu tăng cường bằng polymer sử dụng mô hình sandpack.
\section{Thành viên:}
\begin{table}[h]
\centering
%\caption{My caption}
%\label{my-label}
\begin{tabular}{|c|c|c|c|c|}
\hline
\textbf{TT} & \textbf{Họ và Tên} & \textbf{Lớp} & \textbf{Chuyên ngành} & \textbf{Email}                                    \\ \hline
1           & Bùi Trọng Nghĩa    & K4KKT        & Khoan - Khai thác     & {\color[HTML]{3531FF} {\ul Nghiabt04@pvu.edu.vn}} \\ \hline
2           & Trương Mạnh Phi    & K4KKT        & Khoan - Khai thác     & {\color[HTML]{3531FF} {\ul Phitm04@pvu.edu.vn}}   \\ \hline
\end{tabular}
\end{table}
\section{Cán bộ hướng dẫn trong trường:}
Thạc sĩ Phạm Hữu Tài - cán bộ Khoa Dầu khí - Trường Đại học Dầu khí Việt Nam.
Chuyên ngành: Kỹ thuật dầu khí.
\section{Cán bộ hướng dẫn ngoài trường:}
\section{Lĩnh vực thực hiện:}
Khai thác dầu khí - thu hồi dầu tăng cường.
\section{Thời gian:}
Bắt đầu: 12/3/2018		Kết thúc:15/5/2018
\section{Mục tiêu:}
	Thông qua sử dụng mô hình sandpack trong việc tối ưu hóa tỉ số độ linh động để nâng cao hiệu suất thu hồi dầu tăng cường bằng polymer.


\section{Nội dung đồ án:}
	\begin{enumerate}
		\item[] \textbf{Chương 1: Tổng quan}
		
		\item[] \textbf{Chương 2: Cơ sở lý thuyết}
		\begin{enumerate}
			\item[2.1] Thu hồi dầu tăng cường
			\item[2.2] Polymer Flooding trong thu hồi dầu tăng cường
				\begin{enumerate}
					\item[2.2.1] Hiệu suất đẩy
					\item[2.2.2] Hiệu suất quét
				\end{enumerate}
			\item[2.3] Tỉ số độ linh động
				\begin{enumerate}
					\item[2.3.1] Tỉ số độ linh động trong thu hồi dầu tăng cường
					\item[2.3.2] Những yếu tố ảnh hưởng tới tỉ số độ linh động
					\item[2.3.3] Kiểm soát tỉ số độ linh động
				\end{enumerate}
			\item[2.4] Mô hình sandpack
				\begin{enumerate}
					\item[2.4.1] Mô hình sandpack 1 chiều
					\item[2.4.2] Mô hình sandpack 2 chiều
					%\item[2.4.3] 
				\end{enumerate}
			\item[2.5] Tối ưu hóa tỉ số độ linh động 
				%\begin{enumerate}
					%\item[2.5.1] Mô hình sandpack
					%\item[2.5.2]
				%\end{enumerate}
		\end{enumerate}
		\item[] \textbf{Chương 3: Cases study}
		\item[] \textbf{Chương 4: Kết luận và kiến nghị}
			\begin{enumerate}
				\item[4.1] Kết luận
				\item[4.2] Kiến nghị
			\end{enumerate}
	\end{enumerate}
\section{Phương pháp thực hiện:}
	\begin{enumerate}
		\item[-] Thu thập, nghiên cứu tài liệu
		\item[-] Phân tích, thống kê kết quả số liệu.
	\end{enumerate}
	
\hspace{75pt}
\textbf{Giảng viên hướng dẫn}
\hspace{100pt} \textbf{Đại diện tác giả đồ án}
\newline
\hspace*{88pt} \textit{(Ký, ghi rõ họ tên)}
\hspace{125pt} \textit{(Ký, ghi rõ họ tên)}
\end{document}