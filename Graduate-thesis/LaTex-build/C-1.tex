\immediate\write18{makeindex C-1.nlo -s nomencl.ist -o C-1.nls}
\documentclass[12pt,a4paper]{report}
\usepackage[utf8]{inputenc}
\usepackage{vietnam}
\usepackage[left=3cm, right=2.2cm, top=2.5cm, bottom=2.5cm]{geometry}
\usepackage{amsmath}
\usepackage{float}
\usepackage{amssymb} 
\usepackage{graphicx} 
\usepackage[hidelinks, unicode]{hyperref}
\usepackage[labelsep=period]{caption}
\usepackage[table]{xcolor}
\usepackage{titletoc}
\usepackage{etoc}
\usepackage{mathptmx}
\usepackage{sectsty}
\usepackage{multirow}
\usepackage{booktabs, tabularx}
\usepackage{courier}
\usepackage{subfig}
\usepackage{nomencl}
\usepackage{titlesec}
\usepackage{enumitem}
\usepackage{anyfontsize}
\usepackage[fontsize=13pt]{scrextend}
\newcommand{\code}[1]{\texttt{#1}}
\renewcommand{\baselinestretch}{1.5}
\renewcommand{\nomname}{Danh mục ký hiệu và viết tắt}


\makenomenclature
\makeatletter
\titlecontents{chapter}[3cm] % <-- seems to set some specific left margin
{\color{black}\bfseries\addvspace{3mm}}
{\makebox[0cm][r]{\MakeUppercase\@chapapp\hspace{.5em}\thecontentslabel\hspace{0.75cm}}}
{} %     ^^^ pretendously zero width box puts its contents in the left margin
{\hfill\makebox[-2cm]{\thecontentspage}}  % 3cm = twice 1.5cm
\chapternumberfont{\Large}
\chaptertitlefont{\Large}

\titleformat{\chapter}[hang] 
{\normalfont\fontsize{14}{15}\bfseries}{CHƯƠNG \thechapter.}{1em}{} 
\titlespacing*{\chapter}{0pt}{-7pt}{7pt}

\titleformat{\section}
{\normalfont\fontsize{13}{15}\bfseries}{\thesection.}{1em}{}
\titlespacing*{\section}{0pt}{-5pt}{-6pt}  

\titleformat{\subsection}
{\normalfont\fontsize{13}{15}\bfseries\itshape}{\thesubsection.}{1em}{}
\titlespacing*{\subsection}{0pt}{-20pt}{-6pt}

\titleformat{\subsubsection}
{\normalfont\fontsize{13}{15}\itshape}{\thesubsubsection.}{1em}{}
\titlespacing*{\subsubsection}{0pt}{-20pt}{-6pt}


\newcommand{\subsubsubsection}[1]{\paragraph{#1}\mbox{}\\}


\setlist[itemize]{itemsep=-0.3em, topsep=0pt}
\setlength\parindent{0pt}
\setlength{\parskip}{10pt}
\setcounter{secnumdepth}{4}
\setcounter{tocdepth}{4}
\geometry{letterpaper}

% \title{\textbf{ĐỒ ÁN CÔNG NGHỆ MỎ}}
% \author{Bùi Trọng Nghĩa\\Diệp Công Trứ}

\begin{document}
\pagenumbering{gobble}
% \thispagestyle{empty}
\clearpage

\pdfbookmark{\contentsname}{content}
% \maketitle

\tableofcontents
\addcontentsline{toc}{section}{MỤC LỤC}

\listoffigures
\addcontentsline{toc}{section}{DANH SÁCH HÌNH VẼ}

\listoftables
\addcontentsline{toc}{section}{DANH SÁCH BẢNG BIỂU}

\printnomenclature
\addcontentsline{toc}{section}{DANH MỤC KÝ HIỆU VÀ VIẾT TẮT}

\clearpage
\pagenumbering{arabic}
\newpage

\chapter{CƠ SỞ LÝ THUYẾT}
\section{Các đặc tính dự báo}

\section{Phương pháp dự báo bằng mạng neuron nhân tạo}

\section{Các phương pháp khai thác nhân tạo}
Thông thường, trong giai đoạn đầu của một giếng khai thác dòng lưu chất từ vỉa sẽ khai thác nhờ áp suất vỉa (nếu áp suất vỉa lớn hơn tổng áp suất mất mát của hệ thống khai thác), giai đoạn này được gọi là khai thác tự phun. Nếu như áp suất vỉa không còn đủ lớn để có thể thắng được áp suất mất mát, dòng chảy trong giếng sẽ ``chết'' và giếng không thể tiếp tục khai thác tự phun được nữa. Hai nguyên nhân chính dẫn đến dòng trong giếng bắt đầu ``chết'' chính là:
	\begin{itemize}
		\item Áp suất dòng vào ở đáy giếng giảm xuống thấp hơn tổng áp suất mất mát của giếng,
		\item Tổng áp suất mất mát của giếng lớn hơn áp suất đáy giếng nhỏ nhất để có thể đưa được dòng lưu chất lên bề mặt.
	\end{itemize}
Trường hợp đầu tiên xảy ra khi lưu chất được khai thác khỏi vỉa sẽ dẫn tới áp suất của vỉa suy giảm theo thời gian cho tới khi không còn đủ lớn để thắng được áp suất mất mát của hệ thống khai thác. Đối với trường hợp thứ hai, thường do tăng mức độ cản trở dòng chảy trong hệ thống, có hai nguyên nhân dẫn đến điều này:
	\begin{itemize}
		\item Tỉ trọng của dòng lưu chất tăng do lượng khí đồng hành giảm,
		\item Các vấn đề cơ học như kích thước tubing nhỏ, lắng đọng cặn ...
	\end{itemize}
Khai thác nhân tạo được thực hiện khi dòng chảy trong giếng bắt đầu ``chết'' hoặc tăng lưu lượng khai thác của giếng. Hai phương pháp phổ biến nhất thường được lựa chọn để thực hiện khai thác nhân tạo là bơm điện li tâm chìm (ESP)\nomenclature{ESP}{Electrical Submersible Pump}; một máy bơm sẽ được đặt xuống lòng giếng dưới mực lưu chất trong giếng, bơm tăng áp suất để thắng được tổn thất áp suất trong hệ thống; và bơm ép khí từ bề mặt vào trong tubing khai thác để dòng lưu chất dễ dàng phun lên bề mặt hơn (gas lift). Ngoài ra vẫn còn một số phương pháp khai thác nhân tạo khác nhưng không được sử dụng phổ biến.\\

\subsection{Bơm điện li tâm chìm}
\subsubsection{Ứng dụng}
Được đưa vào ứng dụng trong thực tiễn từ năm 1920 tại mỏ Oklahoma, ESP có thể nâng lưu lượng khai thác thêm tới 1000 thùng trên ngày, gấp khoảng 2 đến 3 lần bơm cần (rod pump).

Ngày nay, bơm điện li tâm chìm thường được ứng dụng trong bơm ép nước (trên bờ), khai thác trên biển hoặc trong những trường hợp có sẵn nguồn điện và khai thác với lưu lượng lớn. ESP có thể được lắp đặt ở độ sâu 1,000 ft đến 10,000 ft đồng thời tăng lưu lượng khai thác thêm từ 200 bbl/d đến 20,000 bbl/d. Kỉ lục độ sâu và lưu lượng có thể sử dụng ESP là 15,000 ft và 30,000 bbl/d.

\subsubsection{Ưu điểm và hạn chế}

Một vài ưu điểm của bơm điện li tâm chìm có thể kể đến như sau:
	\begin{itemize}
		\item Thích hợp với khai thác lưu lượng lớn,
		\item Hiệu suất cao,
		\item Sử dụng tốt trong giếng khoan định hướng,
		\item Có thể sử dụng trong điều kiện khu dân cư,
		\item Thích hợp với hoạt động khai thác trên biển,
		\item Dễ dàng sử lý ăn mòn và lắng đọng.
	\end{itemize}
ESP cũng có một vài hạn chế như:
	\begin{itemize}
		\item Phụ thuộc vào nguồn cung cấp năng lượng,
		\item Cần được vận hành trong điều kiện ổn định,
	\end{itemize}

\section{Đường cong đặc tính gas lift}

\section{Giải thuật di truyền}
\subsection{Nguyên lý giải thuật}

\subsection{Bài toán tối ưu}

\section{Thuật toán}

\end{document}