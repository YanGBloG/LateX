\immediate\write18{makeindex C-3.nlo -s nomencl.ist -o C-3.nls}
\documentclass[12pt,a4paper]{report}
\usepackage[utf8]{inputenc}
\usepackage{vietnam}
\usepackage[left=3cm, right=2.2cm, top=2.5cm, bottom=2.5cm]{geometry}
\usepackage{amsmath}
\usepackage{float}
\usepackage{amssymb} 
\usepackage{graphicx} 
\usepackage[hidelinks, unicode]{hyperref}
\usepackage[labelsep=period]{caption}
\usepackage[table]{xcolor}
\usepackage{titletoc}
\usepackage{etoc}
\usepackage{mathptmx}
\usepackage{sectsty}
\usepackage{multirow}
\usepackage{booktabs, tabularx}
\usepackage{courier}
\usepackage{subfig}
\usepackage{nomencl}
\usepackage{titlesec}
\usepackage{enumitem}
\usepackage{anyfontsize}
\usepackage[fontsize=13pt]{scrextend}
\newcommand{\code}[1]{\texttt{#1}}
\renewcommand{\baselinestretch}{1.5}
\renewcommand{\nomname}{Danh mục ký hiệu và viết tắt}


\makenomenclature
\makeatletter
\titlecontents{chapter}[3cm] % <-- seems to set some specific left margin
{\color{black}\bfseries\addvspace{3mm}}
{\makebox[0cm][r]{\MakeUppercase\@chapapp\hspace{.5em}\thecontentslabel\hspace{0.75cm}}}
{} %     ^^^ pretendously zero width box puts its contents in the left margin
{\hfill\makebox[-2cm]{\thecontentspage}}  % 3cm = twice 1.5cm
\chapternumberfont{\Large}
\chaptertitlefont{\Large}

\titleformat{\chapter}[hang] 
{\normalfont\fontsize{14}{15}\bfseries}{CHƯƠNG \thechapter.}{1em}{} 
\titlespacing*{\chapter}{0pt}{-7pt}{7pt}

\titleformat{\section}
{\normalfont\fontsize{13}{15}\bfseries}{\thesection.}{1em}{}
\titlespacing*{\section}{0pt}{-5pt}{-6pt}  

\titleformat{\subsection}
{\normalfont\fontsize{13}{15}\bfseries\itshape}{\thesubsection.}{1em}{}
\titlespacing*{\subsection}{0pt}{-20pt}{-6pt}

\titleformat{\subsubsection}
{\normalfont\fontsize{13}{15}\itshape}{\thesubsubsection.}{1em}{}
\titlespacing*{\subsubsection}{0pt}{-20pt}{-6pt}


\newcommand{\subsubsubsection}[1]{\paragraph{#1}\mbox{}\\}


\setlist[itemize]{itemsep=-0.3em, topsep=0pt}
\setlength\parindent{0pt}
\setlength{\parskip}{10pt}
\setcounter{secnumdepth}{4}
\setcounter{tocdepth}{4}
\geometry{letterpaper}

% \title{\textbf{ĐỒ ÁN CÔNG NGHỆ MỎ}}
% \author{Bùi Trọng Nghĩa\\Diệp Công Trứ}

\begin{document}
\pagenumbering{gobble}
% \thispagestyle{empty}
\clearpage

\pdfbookmark{\contentsname}{content}
% \maketitle

\tableofcontents
\addcontentsline{toc}{section}{MỤC LỤC}

\listoffigures
\addcontentsline{toc}{section}{DANH SÁCH HÌNH VẼ}

\listoftables
\addcontentsline{toc}{section}{DANH SÁCH BẢNG BIỂU}

\printnomenclature
\addcontentsline{toc}{section}{DANH MỤC KÝ HIỆU VÀ VIẾT TẮT}

\clearpage
\pagenumbering{arabic}
\newpage

\chapter{TRIỂN KHAI MÔ HÌNH}
\begin{center}
	\centering
	A. XÂY DỰNG MÔ HÌNH DỰ BÁO THEO THỜI GIAN
\end{center}
\section{Tiền xử lý dữ liệu}
Tiền xử lý dự liệu là một bước cực kì quan trọng để có thể đưa ra đươc một kết quả tốt nhất, đặc biệt là đối với một bài toán thời gian. Tại bước này những dữ liệu ngoại lai, nhiễu, lặp, hoặc dữ liệu thiếu trong quá trình thu thập dữ liệu sẽ được xử lý nhằm cải thiện chất lượng dữ liệu và chất lượng kết quả.

Một vài kĩ thuật khác để tiền xử lý dữ liệu như tích hợp, biến đổi và thu giảm dữ liệu. Tuy nhiên, những kĩ thuật này yêu cầu có nguồn dữ liệu kích thước lớn, do đó phạm vi luận án sẽ không nhắc đến các kĩ thuật này.





\end{document}