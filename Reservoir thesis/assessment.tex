\documentclass[12pt,a4paper]{report}
\usepackage[utf8]{inputenc}
\usepackage{vietnam}
\usepackage[left=3cm, right=2.2cm, top=2.5cm, bottom=2.5cm]{geometry}
\usepackage{amsmath}
\usepackage{float}
\usepackage{amssymb} 
\usepackage{graphicx} 
\usepackage[hidelinks, unicode]{hyperref}
\usepackage[labelsep=period]{caption}
\usepackage[table]{xcolor}
\usepackage{titletoc}
\usepackage{etoc}
\usepackage{mathptmx}
\usepackage{sectsty}
\usepackage{multirow}
\usepackage{booktabs, tabularx}
\usepackage{courier}
\usepackage{subfig}
\usepackage{nomencl}
\usepackage{titlesec}
\usepackage{enumitem}
\usepackage{anyfontsize}
\usepackage[fontsize=13pt]{scrextend}
\newcommand{\code}[1]{\texttt{#1}}
\renewcommand{\baselinestretch}{1.5}

\setlength\parindent{0pt}

\begin{document}
\pagenumbering{gobble}

\begin{center}
	\centering
	\text{ĐỒ ÁN ĐƯỢC HOÀN THÀNH TẠI}\\ 
	\textbf{TRƯỜNG ĐẠI HỌC DẦU KHÍ VIỆT NAM}
\end{center}
Người hướng dẫn chính:..................................................\\
(\textit{Ghi rõ họ, tên, học hàm, học vị})\\
\newline
Người hướng dẫn phụ (\textit{Nếu có}):......................................\\
(\textit{Ghi rõ họ, tên, học hàm, học vị})\\
\newline
Người chấm phản biện:....................................................\\
(\textit{Ghi rõ họ, tên, học hàm, học vị})\\
\newline
\newline
\newline
\newline
\newline
\newline
Đồ án được bảo vệ tại:
\begin{center}
	\centering
	\textbf{HỘI ĐỒNG CHẤM ĐỒ ÁN MÔN HỌC}\\
	\textbf{TRƯỜNG ĐẠI HỌC DẦU KHÍ VIỆT NAM}\\
	Ngày .... tháng .... năm ....
\end{center}
\newpage

\begingroup
\fontsize{10pt}{12pt}\selectfont
TRƯỜNG ĐẠI HỌC DẦU KHÍ VIỆT NAM \hspace*{1.5cm} \textbf{CỘNG HÒA XÃ HỘI CHỦ NGHĨA VIỆT NAM}\\
\hspace*{1.7cm}\underline{\textbf{KHOA DẦU KHÍ}} \hspace*{4.4cm} \underline{\textbf{Độc Lập - Tự Do - Hạnh Phúc}}
\endgroup

\begin{center}
	\centering
	\textbf{NHIỆM VỤ ĐỒ ÁN MÔN HỌC}
\end{center}

\textbf{Họ và Tên sinh viên thực hiện:}
	\begin{itemize}
		\item Bùi Trọng Nghĩa - MSSV: 04PET110011
		\item Diệp Công Trứ - MSSV: 04PET110017
	\end{itemize}
\textbf{Ngành:} Khoan - Khai thác dầu khí. \hspace*{3cm}\textbf{Lớp:} K4KKT\\
\textbf{1. Tên đồ án môn học:} Tối ưu hóa khai thác cho giếng dầu bằng phương pháp phân tích điểm nút.\\
\textbf{2. Nhiệm vụ:} \\
\hspace*{1cm}- Tìm hiểu về phương pháp phân tích điểm nút\\
\hspace*{1cm}- Các phương trình đặc tính giếng\\
\hspace*{1cm}- Ứng xử của dòng chảy trong ống\\
\hspace*{1cm}- Tối ưu hóa khai thác cho một giếng dầu\\
\hspace*{1cm}- Xây dựng mô hình phân tích điểm nút để thực hiện tối ưu hóa khai thác cho giếng X.\\
\textbf{3. Ngày giao Đồ án môn học: }20/09/2018\\
\textbf{4. Ngày hoàn thành Đồ án môn học: }17/12/2018\\
\textbf{5. Người hướng dẫn: }
	\begin{itemize}
		\item ThS. Nguyễn Viết Khôi Nguyên
		\item ThS. Phạm Hữu Tài
	\end{itemize}
\begin{flushright}
Bà Rịa - Vũng Tàu, ngày \ldots tháng \ldots năm \ldots
\end{flushright}
\begingroup
\fontsize{12pt}{12pt}\selectfont
\textbf{HIỆU TRƯỞNG} \hspace{30pt} \textbf{TRƯỞNG PHÒNG ĐÀO TẠO} \hspace{30pt} \textbf{NGƯỜI HƯỚNG DẪN}\\
(Ký, ghi rõ họ tên) \hspace{60pt} (Ký, ghi rõ họ tên) \hspace{80pt} (Ký, ghi rõ họ tên)
\endgroup

\newpage

\begingroup
\fontsize{10pt}{12pt}\selectfont
TRƯỜNG ĐẠI HỌC DẦU KHÍ VIỆT NAM \hspace*{1.5cm} \textbf{CỘNG HÒA XÃ HỘI CHỦ NGHĨA VIỆT NAM}\\
\hspace*{1.7cm}\underline{\textbf{KHOA DẦU KHÍ}} \hspace*{4.4cm} \underline{\textbf{Độc Lập - Tự Do - Hạnh Phúc}}
\endgroup

\begin{center}
	\centering
	\textbf{PHIẾU NHẬN XÉT ĐỒ ÁN MÔN HỌC}\\
	\textbf{(Giáo viên ghi nhận xét của mình, bằng tay, vào phần này)}
\end{center}
\begin{enumerate}
	\item[1.] Về hình thức kế cấu Đồ án:\\.....................................................................................................................................\\.....................................................................................................................................
	\item[2.] Về nội dung: 
	\begin{enumerate}
		\item[2.1] Nhận xét về phần tổng quan tài liệu:
	\end{enumerate}
	\item[] ....................................................................................................................................\\....................................................................................................................................
	\begin{enumerate}
		\item[2.2] Nhận xét về phương pháp nghiên cứu:
	\end{enumerate}
	\item[] ....................................................................................................................................\\....................................................................................................................................
	\begin{enumerate}
		\item[2.3] Nhận xét về kết quả đạt được:
	\end{enumerate}
	\item[] ....................................................................................................................................
	\begin{enumerate}
		\item[2.4] Nhận xét phần kết luận:
	\end{enumerate}
	\item[] ....................................................................................................................................
	\begin{enumerate}
		\item[2.5] Những thiếu sót và tồn tại của Đồ án:
	\end{enumerate}
	\item[] ....................................................................................................................................
\end{enumerate}
\textbf{Điểm:}....................................................(Ghi bằng chữ)
\begin{flushright}
Bà Rịa - Vũng Tàu, ngày \ldots tháng \ldots năm \ldots \\
\end{flushright}
\hspace{300pt} \textbf{NGƯỜI HƯỚNG DẪN}\\
\hspace*{316pt} (Ký, ghi rõ họ tên)

\end{document}