\PassOptionsToPackage{unicode=true}{hyperref}
\documentclass[11pt]{beamer}

% For more themes, color themes and font themes, see:
% http://deic.uab.es/~iblanes/beamer_gallery/index_by_theme.html
%
\mode<presentation>
{
  \usetheme{Madrid}       % or try default, Darmstadt, Warsaw, ...
  \usecolortheme{default} % or try albatross, beaver, crane, ...
  \usefonttheme{serif}    % or try default, structurebold, ...
  \setbeamertemplate{navigation symbols}{}
  \setbeamertemplate{caption}[numbered]
} 

\usepackage[vietnam]{babel}
\usepackage[utf8]{inputenc}
\usepackage{sansmathaccent}
\usepackage{vietnam}
\usepackage{pgfpages}
\usepackage{amsmath}
\usepackage{amsfonts}
\usepackage{amssymb}
\usepackage{graphics} % thư viện để hiển thị ảnh
\usepackage{caption} % thư viện để đặt caption
\usepackage{booktabs} % thư viện để vẽ table
\usepackage{lmodern}
\usepackage{hyperref}
%\pdfmapfile{+sansmathaccent.map}
%\pgfpagesuselayout{resize to}[%
%  physical paper width=8in, physical paper height=6in]

% Here's where the presentation starts, with the info for the title slide
\title[Japan Vietnam Petroleum Company]{Intership Report}
\author{Trọng Nghĩa}
\institute{PVU}
\date{\today}

\begin{document}
\begin{frame}
  \titlepage
\end{frame}

% tạo toc
\begin{frame}{Nội dung}
  \tableofcontents
\end{frame}

\section{Tổng quan}
\begin{frame}{Tổng quan}
	\begin{itemize}
		\item Phân tích thử vỉa với phần mềm PanSystem
			\begin{itemize}
				\item[-] Phân tích thử vỉa
				\item[-] Thực hành với phần mềm PanSystem
			\end{itemize}
		\item Phân tích cân bằng vật chất với phần mềm MBAL
			\begin{itemize}
				\item[-] Phân tích cân bằng vật chất
				\item[-] Thực hành với phần mềm MBAL
			\end{itemize}
		\item Mô phỏng vỉa với ECLIPSE
			\item[-] Mô phỏng vỉa
			\item[-] Thực hành với ECLIPSE
	\end{itemize}
\end{frame}
\section{Nguyên lý cơ bản}
\subsection{Phân tích thử vỉa với phần mềm PanSystem}
\begin{frame}{Phân tích thử vỉa}
Mục tiêu:
	\begin{itemize}
		\item Độ thấm
		\item Mức độ bất đồng nhất của vỉa (Vỉa hai độ rỗng, đa lớp, vỉa phức hợp...)
		\item Hệ số nhiễm bẩn 
		\item Hình dạng vỉa và kích thước vỉa
		\item Áp suất vỉa ban đầu
	\end{itemize}
\end{frame}
\begin{frame}{Phân tích thử vỉa}
Phương trình phân tán \eqref{eqn:diffusion}:
	\begin{equation}\label{eqn:diffusion}
		\dfrac{1}{r}\dfrac{\partial}{\partial r}\dfrac{r\partial p}{\partial r} = \dfrac{\phi\rho c}{k}\dfrac{\partial p}{\partial t}
	\end{equation}
\end{frame}
\begin{frame}{PanSystem}
Ứng dụng:
	\begin{itemize}
		\item Phân tích thử vỉa, khớp lịch sử bằng các phương pháp \textit{Số học, Giải tích}.
		\item Dự đoán khai thác dài hạn
		\item Thiết kế thử vỉa
	\end{itemize}
\end{frame}
\subsection{Phân tích cân bằng vật chất với phần mềm MBAL}
\begin{frame}{Cân bằng vật chất}
Phương trình cơ bản:\\
\hspace*{1cm} Lượng lưu chất vỉa ban đầu\\
\hspace*{2cm} = Lượng lưu chất khai thác + Lượng lưu chất còn lại\\
Dạng đơn giản:
	\begin{equation}\label{eqn:straight_line}
		F = N \times{E_t} + W_e
	\end{equation}
Với:\\
\hspace*{1cm} $F$ là lượng khai thác được\\
\hspace*{1cm} $N$ là lượng HC (Hydrocarbon) ban đầu trong vỉa\\
\hspace*{1cm} $E_t$ là lượng lưu chất giãn nở\\
\hspace*{1cm} $W_e$ là lượng nước thâm nhập từ tầng nước đáy.
\end{frame}
\begin{frame}{Cân bằng vật chất}
Ứng dụng:
	\begin{itemize}
		\item Tính toán định lượng các thông số vỉa (HC, kích thước mũ khí...)
		\item Xác định tầng nước đáy, góc bao phủ
		\item Xác định cơ chế khai thác.
	\end{itemize}
\end{frame}
\begin{frame}{Phần mềm MBAL}
Chức năng:
	\begin{itemize}
		\item Phân tích cân bằng vật chất
		\item Phân tích đường công suy giảm
		\item Mô phỏng Monte Carlo
		\item Phân chia rạn giới vỉa
		\item Phân tích vỉa đa lớp
		\item Phân tích vỉa khí có độ thấm thấp.
	\end{itemize}
\end{frame}
\subsection{Mô phỏng vỉa với phần mềm ECLIPSE	}
\begin{frame}{Mô phỏng vỉa}
Các phương trình cơ bản:
	\begin{itemize}
		\item Định luật Darcy \eqref{eqn:darcy_law} (bỏ qua ảnh hưởng của trọng lực)
		\item Phương trình cân bằng khối lượng \eqref{eqn:mass_balance}
		\item Phương trình mô phỏng dòng chảy \eqref{eqn:flow_equa} (bỏ qua ảnh hưởng của trọng lực).
	\end{itemize}
	\begin{equation}\label{eqn:darcy_law}
		q = -\dfrac{k}{\mu}\nabla{P}
	\end{equation}
	\begin{equation}\label{eqn:mass_balance}
		-\nabla{M}=\dfrac{\partial}{\partial t}(\phi\rho)+Q
	\end{equation}
	\begin{equation}\label{eqn:flow_equa}
		\nabla[\lambda(\nabla{P}-\gamma\nabla{z})]=\dfrac{\partial}{\partial{t}}\dfrac{\phi}{\beta}+\dfrac{Q}{\rho}
	\end{equation}
\end{frame}

\end{document}